
\documentclass[a4paper,12pt]{article} % тип документа


% Русский язык
\usepackage[T2A]{fontenc} % кодировка
\usepackage[utf8]{inputenc} % кодировка исходного текста
\usepackage[english,russian]{babel} % локализация и переносы


% Математика
\usepackage{amsmath,amsfonts,amssymb,amsthm,mathtools}


\usepackage{wasysym}

%Заговолок
\author{Талашкевич Даниил Александрович}

\title{Домашнее задание по дискретному анализу. Неделя 3. Математические утверждения и доказательства.}

\date{\today}

\begin{document}

\maketitle
\thispagestyle{empty}

\newpage
\setcounter{page}{1}
\begin{center}
\itshape
\bfseries
{ \Large Problems:}
\end{center}

{\bf 1.} Убедитесь в истинности утверждения (при произвольных A и B):
$(A \rightarrow B) \vee (B \rightarrow A).$ (1)
Зафиксируем произвольную параболу. Пусть A — утверждение «ветви
параболы направлены вверх», а B =«парабола пересекает 0 (прямую
y = 0)». Проследите за следующими рассуждениями. Утверждение
«если ветви параболы направлены вверх, то парабола пересекает 0»,
очевидно, ложно; тогда истинным должно быть утверждение «если
парабола пересекает 0, то ветви параболы направлены вверх», но оно
также ложно. То есть оба утверждения в дизъюнкции (1) ложны (при
некотором выборе утверждений A и B), но сама дизъюнкция истинна!
Найдите ошибку в рассуждениях.



\begin{center}
\bfseries
{\Large Решение: }
\end{center}

Утверждение "ветви параболы направлены вверх"\hspace{1pt} обозначим за $A$, а "парабола пересекает 0"\hspace{1pt} за $B$. \\
В рассуждениях сказано, что если ветви параболы направлены вверх, то парабола пересекает 0 -- ложное утверждение, то есть $A\rightarrow B = 0$, что возможно, только если $A = 1$ и $B = 0$, значит парабола никогда не пересекает ноль, что и является ошибкой в рассуждениях -- если ветви параболы направлены вверх, то возможна ситуация, когда парабола пересекает ноль.\\
Если же парабола, ветви которой направлены вверх, действительно не пересекает 0, то истинно утверждение "если ветви параболы направлены вниз, то она пересекает 0". Рассуждения симметричны в обе стороны. Тогда либо $A\rightarrow B = 1$, либо $B\rightarrow A = 1$, значит всегда выполняется формула
$(B\rightarrow A = 1) \vee (A\rightarrow B = 1)$.

\begin{flushright}
\begin{large}
\textbf {Ответ: доказано.}
\end{large}
\end{flushright}

{\bf 2.} В доме живут A, его жена B и их дети C, D, E, при этом справедливы
следующие утверждения:
1) если A смотрит телевизор, то и B смотрит телевизор;
2) хотя бы один из D и E смотрит телевизор;
3) ровно один из B и C смотрит телевизор;
4) C и D либо оба смотрят, либо оба не смотрят телевизор;
5) если E смотрит телевизор, то A и D тоже смотрят телевизор.
Кто смотрит и кто не смотрит телевизор?

\begin{center}
\bfseries
\newpage
{\Large Решение: }
\end{center}
Согласно условиям можно записать:
  \[\begin{cases}
    A \rightarrow B = 1\\
    D \vee E = 1\\
    B \oplus C = 1\\
    C \leftrightarrow D = 1\\
    E \rightarrow (A \wedge D) = 1
 \end{cases}\]\\
Рассмотрим случай, когда $E = 1$, тогда $A = D = 1$ (последнее условие). Из 1-ого условия следует, что $B = 1$, тогда из 3-его условия $C = 0$. Получили, что не выполняется 4-ое условие -- противоречие. Итак, $E = 0$.\\
Теперь система выглядит так:
 
   \[\begin{cases}
    A \rightarrow B = 1\\
    D = 1\\
    B \oplus C = 1\\
    C \leftrightarrow D = 1\\
 \end{cases}\]\\
 Здесь из 4-ого условия следует, что $C = 1$, из 3-его условия -- $B = 0$, тогда из 1-ого -- $A = 0$. \\
 Итак, $A = 0$, $B = 0$, $C = 1$, $D = 1$, $E = 0$. Телевизор смотрят только $C$ и $D$.

\begin{flushright}
\begin{large}
\textbf {Ответ: только C и D.}
\end{large}
\end{flushright}

       
{\bf 3.} Докажите, используя контрапозицию, что если $x^2 - 6x + 5$ чётно, то x нечётно; здесь $x \in Z$.
\begin{center}
\bfseries
{\Large Решение: }
\end{center}
Предположим, что $x^2 - 6x + 5$ нечётно, тогда чётно $x^2 - 6x = x(x-6)$. То есть либо чётно $x$, либо $x-6$. Если чётно $x$, то чётно и $x-6$ и наоборот, значит $x(x-6)$ всегда чётно, если чётно $x$. Значит если нечётно  $x^2 - 6x + 5$, то чётно $x$. Отсюда получаем, что если чётно $x^2 - 6x + 5$, то нечётно $x$.


 
\begin{flushright}
\begin{large}
\textbf {Ответ: доказано.}
\end{large}
\end{flushright}


{\bf 4.} Докажите, что произведение положительного рационального числа и иррационального числа — иррациональное число.
\begin{center}
\bfseries
{\Large Решение: }
\end{center}
Пусть $a$ - рациональное число, а $b$ - иррациональное число. Пусть число $ab$ -- рациональное. Пусть числа $a = p_1/q_1$ и $ab = p_2/q_2$ - несократимые дроби, где $p_1, q_1, p_2, q_2$ -- целые числа, тогда $b = (ab)/a = p_2q_1/q_2p_1 $ -- дробь с целыми числами $p_2q_1$ и $q_2p_1$, тогда $b$ -- рациональное число. Но по предположению $b$ -- иррациональное число -- противоречие. \\
Итак, произведение положительного рационального числа и иррационального числа -- иррациональное число.


\begin{flushright}
\begin{large}
\textbf {Ответ: доказано.}
\end{large}
\end{flushright}

{\bf 5.} Про непустые попарно несовпадающие множества A, B и C известно,
что $C \setminus A \subseteq B $ и $C \setminus B \subseteq A$ . Возможно ли, что $B = A \cap C?$


\begin{center}
\bfseries
{\Large Решение: }
\end{center}
Сразу подставим $B$ в условия, получим систему:
   \[\begin{cases}
  C \setminus A \subseteq (A \cap C)\\
  C \setminus (A \cap C) \subseteq A 
 \end{cases}\]\\ 
 Очевидно, что $C \setminus A \nsubseteq A$, при этом $(A \cap C)\subseteq A$, но из первого условия системы следует, что $C \setminus A \subseteq (A \cap C) \subseteq A $ -- противоречие.
\[C \setminus (A \cap C) = C \cap \overline{(A \cap C)} = C \cap (\overline{A} \cup \overline{C}) = (C \cap \overline{A}) \cup \varnothing = C \cap \overline{A} = C \setminus A \nsubseteq A\]\\
По условию же полученное сверху равенство не выполняется -- второе противоречие.\\
Итак, такая ситуация невозможна.

\begin{flushright}
\begin{large}
\textbf {Ответ: невозможна.}
\end{large}
\end{flushright}

{\bf 6.} Докажите равенства:\\
\[a)\ 1 \cdot(n - 1) + 2 \cdot (n - 2) + \dots + (n - 1) \cdot 1 = \frac{(n-1)\cdot n\cdot (n+1)}{6};\]\\
\[b)\  \cos{x} + \cos2x + \dots + \cos{nx} = \frac{sin(n+\frac{1}{2})x}{2sin\frac{x}{2}} - \frac{1}{2};\]\\
\newpage
\begin{center}
\bfseries
{\Large Решение: }
\end{center}
a)Перепишем начальное уравнение в виде: $\sum\limits_{i=1}^n i(n-i) = n\sum\limits_{i=1}^n i - \sum\limits_{i=1}^n i^2.$\\
Найдем формулу для $\sum\limits_{i=1}^n i^2:$

$ (1+1)^{3}+(2+1)^{3}+ \dots + (n+1)^{3} $\hspace{25mm} (1)

$  1^3 +2^3 + \dots + n^3 $ \hspace{54.8mm} (2)\\
$ (1) - (2) = (n+1)^3-1^3=3n^3+3n^2+3n.$
А с другой стороны мы можем рассматривать это как из каждого $X_{i}$ элемента (1) отнять $X_{i}$ элемент (2): $(n+1)^3 - n^3 = 3n^2 +3k + 1$. Тогда имеем 
$ 3\cdot(1^2+2^2+\cdots +n^2)+3\cdot(1+2+\cdots + 3)+1(1+1+\cdots+n) =  3n^2+2n+1 \Rightarrow$
$3S_{2}+3S_{1}+n = 3n^3+3n^2+3n \Rightarrow S_{2}= \sum\limits_{i=1}^n i^2 = \frac{n(n+1)(2n+1)}{6}.$\\
Теперь окончательно выразим: $\sum\limits_{i=1}^n i(n-i) = n\sum\limits_{i=1}^n i - \sum\limits_{i=1}^n i^2 = n\cdot \frac{(n+1)}{2}\cdot n -$
$ -\frac{n(n+1)(2n+1)}{6} = \frac{(n-1)n(n+1)}{6}.$\\
Мы доказали для $\forall n \in \mathbb{N}$.\\

б) База индукции n = 1:

\[ \cos{x} = \frac{sin(1+\frac{1}{2})x}{2sin\frac{x}{2}} - \frac{1}{2} \Rightarrow \cos{x} = \frac{\sin{1,5x} - \sin{\frac{x}{2}}}{2\cdot \sin{\frac{x}{2}}} \Rightarrow \cos{x} = \frac{2\cdot \cos{x}\cdot \sin{0,5x}}{2\cdot \sin{0,5x}}.\]\\
Здесь использована формула для разности синусов:
\[ \sin{t} - \sin{s} = 2\cdot \cos{\frac{t+s}{2}}\cdot \sin{\frac{t - s}{2}}. \hspace{47mm} (*) \]
Выполняется.\\
Пусть при $n = k$	 выполняется:

\[ \cos{x} + \cos2x + \dots + \cos{kx} = \frac{sin(k+\frac{1}{2})x}{2sin\frac{x}{2}} - \frac{1}{2};\]\\
Докажем для $n = k + 1:$

\[ \cos{x} + \cos2x + \dots + \cos{(k+1)x} = \frac{sin(k+1+\frac{1}{2})x}{2sin\frac{x}{2}} - \frac{1}{2}.\hspace{8mm} (1)\]
\[ \frac{sin(k+\frac{1}{2})x}{2sin\frac{x}{2}} - \frac{1}{2} + \cos{(k+1)x} = \frac{\cos{\frac{(k+2)x}{2}}\cdot \sin{\frac{(k+1)\cdot x}{2}}}{\sin{0,5x}}.\hspace{12mm} (2)\]\\
Здесь так же была использована формула (*).\\
Для левой части (2):

\[ \frac{\sin{\frac{(2k+1)x}{2}}}{2} - \frac{\sin{\frac{x}{2}}}{2} + \frac{\sin{\frac{(2k+3)x}{2}}}{2} - \frac{\sin{\frac{(2k+1)x}{2}}}{2}.\]\\
Для правой (2):

\[ \frac{\sin{\frac{(k+2+k+1)x}{2}}}{2} - \frac{\sin{\frac{x}{2}}}{2} .\]\\
Отсюда получаем, что $ 0 = 0 .$ Итак, доказано методом математической индукции.

\begin{flushright}
\begin{large}
\textbf {Ответ: а)доказано.\\б)доказано.}
\end{large}
\end{flushright}


{\bf 7.} В зачете участвовало несколько студентов и преподавателей. Известно, что в комнату, где происходил зачет, каждый участник зачета
вошел лишь однажды и что каждый преподаватель поговорил с каждым студентом. Докажите, что в какой-то момент зачета в комнате
присутствовали либо все студенты (и, может быть, кто-то из преподавателей), либо все преподаватели (и, может быть, кто-то из студентов).
\begin{center}
\bfseries
{\Large Решение: }
\end{center}
Рассмотрим случай, когда ни в какой момент времени в аудитории не находились все преподаватели одновременно. Тогда по-любому для того, чтобы в аудиторию вошёл последний преподаватель, должен выйти какой-то другой преподаватель до этого. Но этот другой преподаватель может выйти только в том случае, когда он уже примет всех студентов, а для этого все студенты должны были поприсутствовать в аудитории, к тому же никто оттуда ещё не ушёл, так как они не сдали предмет последнему преподавателю. Значит в этом случае в какой-то момент времени в аудитории присутствуют все студенты.

Теперь рассмотрим случай, когда ни в какой момент времени в аудитории не находились все студенты одновременно. Тогда по аналогии чтобы вошёл последний студент нужно, чтобы какой-то другой студент уже ушёл, но тогда этот студент должен сдать все предметы, откуда следует, что все преподаватели собрались в аудитории, так как они как минимум не приняли одного студента. Значит в этом случае в какой-то момент времени в аудитории присутствуют все преподаватели.



\begin{flushright}
\begin{large}
\textbf {Ответ: доказано.}
\end{large}
\end{flushright}


{\bf 8. }Найдите ошибку в доказательстве по индукции утверждения «все
лошади одного цвета», точнее A(n) =«любые n лошадей одного цвета».
База, A(1), очевидна: одна лошадь одного цвета. Шаг, $A(n) \rightarrow A(n + 1):$
уберём одну лошадь из n+1 и воспользуемся предложением A(n) — получим, что оставшиеся n лошадей одного цвета. Вернём теперь убранную
лошадь, уберём другую лошадь и опять воспользуемся предложением
A(n) — опять получим, что выбранные n лошадей одного цвета, а значит и убранная в начале лошадь того же цвета, что и все остальные.
Итак, мы доказали, что любая n + 1 лошадь одного цвета.
 
\newpage
\begin{center}
\bfseries
{\Large Решение: }
\end{center}
Так как наш шаг индукции не сообразуется с базой,так как он верен лишь при K$\geqslant 2$,а база индукции n = 1. Иными словами при ${\displaystyle K=1}$ получаемые множества оставшихся лошадей не будут пересекаться, и утверждение о равенстве цветов всех лошадей сделать нельзя. То есть чтобы мы могли убирать лошадь и это не было ошибкой, то мы должны доказать базовую индукцию при n = 2, а это не доказывается.
\begin{flushright}
\begin{large}
\textbf {Ответ: ошибка найдена.}
\end{large}
\end{flushright}

{\bf 9. }В прямоугольнике $3\cdot n$ стоят фишки трех цветов, по n штук каждого
цвета. Докажите, что можно переставить фишки в каждой строке так,
чтобы в каждом столбце были фишки всех цветов.


\begin{center}
\bfseries
{\Large Решение: }
\end{center}
Воспользуемся методом математической индукции: база индукции \\($n = 1$) очевидна (у нас только один столбик и как бы мы не расставляли цвета -- у нас будут все цвета).\\
Пусть при $ n = k$ выполняется, тогда докажем для $ n = k+1$ :\\
Пусть в верхней клетке последнего столбца стоит фишка цвета $a$. Разберём два случая:\\
1. В третьей строке есть фишки только одного цвета, отличного от $a$: обозначим этот цвет $c$. Во второй строке должна быть фишка цвета $b$ (в третьей строке их нет, $а$ в первой их меньше $n$), переставим её в последний столбец.\\
2. В третьей строке есть фишки обоих оставшихся цветов. Заметим, что во второй строке есть фишка цвета, отличного от $а$ (фишек цвета $a$ всего $n$, а одна уже занята в первой строке). Пусть это фишка цвета $b$, переставим её в последний столбец. В третьей строке есть фишка третьего цвета $c$, переставим её в последний столбец.\\
  В любом случае в последнем столбце оказались фишки всех трёх цветов. Отбросив его, мы получим таблицу  ($3\cdot k)$,  удовлетворяющую условию задачи. По предположению индукции фишки в ней можно переставить требуемым образом. Тем самым, мы переставили фишки и в этой таблице $3 \cdot (k+1)$.\\
Итак, доказано методом математической индукции.




\begin{flushright}
\begin{large}
\textbf {Ответ: доказано.}
\end{large}
\end{flushright}

\newpage

{\bf 10. }В утверждении ниже встречаются два высказы-
вания, к которым применены некоторые логические связки:
\begin{center}
"за окном идет дождь, то, если за окном не\\
идет дождь, то гипотеза Римана верна".
\end{center} 

Запишите эти два высказывания, а также перепишите утверждение в формульном виде (с переменными и логическими связками) и докажите, что получившаяся формула является тавтологией.

\begin{center}
\bfseries
{\Large Решение: }
\end{center}
1. Обозначим высказывание : "за окном идем дождь "\hspace{1mm} за $A = 1$, тогда высказывание "за окном не идет дождь"\hspace{1mm} за $\overline{A} = 0$ .\\
2. Тогда $A \rightarrow \overline{A} = \overline{A} \vee \overline{A} = \overline{A} = 0$.\\
3. Высказываение "Гипотеза Римана верна" обозначим за $B$. Тогда:\\
$\overline{A} \rightarrow B = 0 \rightarrow B = 1$ при любых значениях $B$. Значит получившаяся формула является тавтологией.




\begin{flushright}
\begin{large}
\textbf {Ответ: доказано. }
\end{large}
\end{flushright}







\end{document}
