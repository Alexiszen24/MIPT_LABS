
\documentclass[a4paper,12pt]{article} % тип документа


% Русский язык
\usepackage[T2A]{fontenc} % кодировка
\usepackage[utf8]{inputenc} % кодировка исходного текста
\usepackage[english,russian]{babel} % локализация и переносы


% Математика
\usepackage{amsmath,amsfonts,amssymb,amsthm,mathtools}


\usepackage{wasysym}

%Заговолок
\author{Талашкевич Даниил Александрович}

\title{Неделя 7.Комбинаторика I. Правила суммы и произведения}

\date{\today}

\begin{document}

\maketitle
\thispagestyle{empty}

\newpage
\setcounter{page}{1}
\begin{center}
\itshape
\bfseries
{ \Large Problems:}
\end{center}

{\bf 1.} Есть 6 кандидатов на 6 вакансий. Сколькими способами можно
заполнить вакансии? (Каждая вакансия должна быть заполнена.)
\begin{center}
\bfseries
{\Large Решение: }
\end{center}

Выбираем первого кандидата, всего это можно сделать 6 способами. Дальше выбираем пятого из пяти оставшихся, всего это можно сделать 5 способами. Продолжая так далее получим $6! = 720$ вариантов.

\begin{flushright}
\begin{large}
\textbf {Ответ: $720$ способов.}
\end{large}
\end{flushright}

{\bf 2.} а) Каких чисел больше среди первого миллиона: тех, в записи кото-
рых есть единица или тех, в записи которых её нет?

б) Тот же вопрос для первых 10 миллионов чисел.
\begin{center}
\bfseries
{\Large Решение: }
\end{center}

а) Рассмотрим числа от 0 до 999.999, представим это число в виде $\overline{a_1a_2a_3a_4a_5a_6}$. Рассмотрим все числа с единицами в записи, всего единиц в таком числа может быть от 1 до 6: 1 единица -- $C^1_6\cdot 9^5$, 2 единицы -- $C^2_5\cdot 9^4 \dots \Rightarrow $ всего чисел с единицами в своей записи: $6\cdot 59049 + 15\cdot 6561 + 20\cdot 729 + 15\cdot 81 + 6\cdot 9 + 1 = 468559 < 500000,$ значит среди первого миллиона чисел с единицей меньше, чем без нее.

б) Аналогично первому пункту имеем: $C^1_7\cdot 9^6 + C^2_7\cdot 9^5  + C^3_7\cdot 9^4 + C^4_7\cdot 9^3 + C^5_7\cdot 9^2 + C^6_7\cdot 9^1 + C^7_7\cdot 9^0= 468559 · 7 + 1 =
7 \cdot 531441 + 21 \cdot 59049 + 35 \cdot 6561 + 35 \cdot 729 + 21 \cdot 81 + 7 \cdot 9 + 1 = 5217031$. Значит среди первых 10 миллионов чисел с единицей больше чем без неё.

\begin{flushright}
\begin{large}
\textbf {Ответ: а) без единиц больше, б) с единицами больше.}
\end{large}
\end{flushright}

{\bf 3.} Найдите вероятность того, что в десятичной записи случайного
шестизначного числа, в записи будет хотя бы две одинаковые цифры?
\begin{center}
\bfseries
{\Large Решение: }
\end{center}

Для начала найдём вероятность того, что в шестизначном числе все цифры разные. Первую цифру в числе можно выбрать $9$ способами (всё, кроме нуля). Вторую цифру можно поставить $9$ способами (всё, кроме первого), третью цифру $8$ способами (всё, кроме первой и второй цифры), и т.д. Значит всего $9 \cdot 9 \cdot 8 \cdot ... \cdot 5$.

Всего есть $9 \cdot 10^5$ (первая цифра не равна нулю) свособов выбрать шестизначное число, значит вероятность того, что все цифры в шестизначном числе различаются, равна
\[P = \frac{9 \cdot 9 \cdot 8 \cdot ... \cdot 5}{9 \cdot 10^5} = \frac{9 \cdot 8 \cdot ... \cdot 5}{10^5}\]

Во всех остальных числах хотя бы две цифры будут совпадать. Полная вероятность выбора шестизначного числа равна $1$, значит вероятность выбора шестизначного числа такого, что хотя бы две цифры совпадают, равна
\[P_0 = 1 - P = 1 - \frac{9 \cdot 8 \cdot ... \cdot 5}{10^5}\]


\begin{flushright}
\begin{large}
\textbf {Ответ: $1 - \frac{9 \cdot 8 \cdot ... \cdot 5}{10^5}$}
\end{large}
\end{flushright}

{\bf 4.} Из 36-карточной колоды карт на стол равновероятно и случайно
выкладывается последовательность из 4 карт. Какова вероятность того,
что две из них красные, а две черные?
\begin{center}
\bfseries
{\Large Решение: }
\end{center}

Всего карт 36, а каждого цвета по 18( всего 4 масти и по 2 масти на 1 цвет, значит 18 карт красного и 18 карт черного цвета). Отсюда вариантов выбрать два красные или 2 черные карты = $C^2_{18}$. В свою очередь всего вариантов выбрать 4 карты $C^4_{36}$. Значит шанс выбрать 4 карты так, чтобы 2 карты из них были красные и 2 черные = $\frac{C^2_{18}\cdot C^2_{18}}{C^4_{36}} = 0,397$.

\begin{flushright}
\begin{large}
\textbf {Ответ: 0,397.}
\end{large}
\end{flushright}

{\bf 5.} Сколько существует 6-значных чисел, в которых чётных и нечётных
цифр поровну?
\begin{center}
\bfseries
{\Large Решение: }
\end{center}

Так как всего 6 цифр, то всего должно быть $3$ чётные и $3$ нечётные цифры.

Выберем первую цифру произвольным образом -- $9$ способов. Значит из пяти оставшихся цифр ровно 2 чётных и 3 нечётных или 3 чётных и 2 нечётных. Для каждой из 5 цифр есть 5 вариантов выбора цифры в зависимости от того, какое число -- чётное или нечётное, также надо учесть, что цифры могут быть переставлены $C^3_5 = C^2_5$ способами.

Значит по итогу получаем в этом случае $9 \cdot 5^5 \cdot C^3_5 = 90 \cdot 5^5$ чисел.


\begin{flushright}
\begin{large}
\textbf {Ответ:  $90 \cdot 5^5$ чисел}
\end{large}
\end{flushright}

{\bf 6.} Сколько существует 7-значных чисел, в которых ровно две четные
цифры и перед каждой четной цифрой обязательно стоит нечетная?
\begin{center}
\bfseries
{\Large Решение: }
\end{center}

Число состоит из $3-x$ нечётных цифр, $2-x$ чётных и $2-x$ нечётных перед
ними.
Объединим чётную и стоящую перед ней нечётную цифру в одну. Тогда можно выбрать все места как они могут распологаться всего способами $4+3+2+1 = 10$ или же нам нужно выбрать $2$ места из $5$ возможных для псевдоцифры: $C^2_5 = 10$. \\
Выбрать 3 цифры из 5 нечётных: $5^3$ способами. Выбрать одну нечётную и одну чётную цифру для псевдоцифры: $5 \cdot 5$ . В итоге выходит таких чисел: $C^2_5 \cdot 5^3 \cdot (5 \cdot 5)^2 = 781250$.

\begin{flushright}
\begin{large}
\textbf {Ответ: 781250.}
\end{large}
\end{flushright}

{\bf 7.} Сколькими способами можно поселить 7 студентов в три комнаты:
одноместную, двухместную и четырехместную?
\begin{center}
\bfseries
{\Large Решение: }
\end{center}

Выберем для начала $4$ человека для проживания в четырёхместной комнате. Нам не важен порядок людей, важен только набор из четырёх людей. Значит количество способов из выбрать равно $C^4_7$. Далее нужно из трёх оставшихся людей выбрать двух для проживания в двухместной комнате -- $C^2_3$ способа. Остался $1$ человек и $1$ комната -- $1$ способ.

Итого получаем $C^4_7 \cdot C^2_3 = 105$  способов расставить $7$ человек в четырёхместную, двухместную и одноместную комнаты.


\begin{flushright}
\begin{large}
\textbf {Ответ: $105$ способов}
\end{large}
\end{flushright}

{\bf 8.} Найдите количество диаметров в полном бинарном дереве ранга n.
\begin{center}
\bfseries
{\Large Решение: }
\end{center}

Если ранг отсчитывается от $ n = 1$, то в самом последнем уровне дерева будет $2^{n-1}$ вершин. Диаметр -- $d = max\  max\ \rho (u,v)$, для $\forall u,v$. Тогда диаметр -- это расстояние между любой нижней точкой из "левой"\hspace{1mm} части дерева и любой нижней точкой из "правой"\  части дерева. Одну такую точку можно выбрать слева $2^{n-1}$ способами и ,соответственно, справа $2^{n-1}$ способами. Значит число диаметров $|d| = (2^{n-1})^2 = 2^{2n-2}$.

\begin{flushright}
\begin{large}
\textbf {Ответ: $|d| = 2^{2n-2}$.}
\end{large}
\end{flushright}

{\bf 9.} Разбиением числа $N$ на $k$ частей называется такая невозрастающая
последовательность положительных целых чисел $\lambda_1 \geqslant \lambda_2 \geqslant \cdots \geqslant \lambda_k$, что
что $\lambda_1 + \lambda_2 +\cdots +\lambda_k = N$. Чего больше, разбиений числа $N$ на не более
чем $k$ слагаемых, или разбиений числа $N + k$ на ровно $k$ слагаемых?

\begin{center}
\bfseries
{\Large Решение: }
\end{center}

Обозначим количество разбиений числа $N$ на $k$ слагаемых за $S(N,k)$. Тогда, согласно условиям, нужно сравнить два числа: количество разбиений числа $N$ на не более
чем $k$ слагаемых, или разбиений числа $N + k$ на ровно $k$ слагаемых:
\[\sum_{i=1}^k S(N,i) \text{ }?\text{ } S(N+k,k)\]

Для решения задачи воспользуемся следующим: требуемые разбиения чисел являются неупорядоченным разбиением числа $n$ на $k$ слагаемых. Если обозначить число $n$ за $n$-элементное множество, то его разбиение равно разбиению множества на $k$ непустых подмножеств, где за подмножества мы обозначим числа, на которые разбивается число $n$. Так мы показали, что требуемые разбиения числа $N$ на $k$ слагаемых соответствует числу Стирлинга II рода: количество неупорядоченных разбиений $n$-элементного множества на $k$ непустых подмножеств.

Для чисел Стирлинга II рода выполняется рекуррентное соотношение, которое мы будем использовать для разбиений:

\[S(N,k) = S(N-1,k-1) + k\cdot S(N-1,k)\]

Из этого соотношения следует, что
  \[S(N + k,k) - S(N+k-1,k-1) = k\cdot S(N+k-1,k)\]
  \[S(N + k-1,k-1) - S(N+k-2,k-2) = (k-1) \cdot S(N+k-2,k-1)\]
  \[...\]
  \[S(N + 1,1) - S(N,0) = S(N,1)\]
Сложим теперь все равенства:
\[S(N + k,k) - S(N,0) = \sum_{i=1}^k i \cdot S(n+(i-1), i)\]
Для граничных условий имеем $S(N,0) = 0$, тогда
\[S(N + k,k) = \sum_{i=1}^k i \cdot S(N+(i-1), i)\]
Значит по условию нужно сравнить две суммы:
\[\sum_{i=1}^k i \cdot S(N+(i-1), i) \text{ }?\text{ } \sum_{i=1}^k S(N,i)\]

Очевидно, что $S(N+(i-1), i) > S(N, i)$, так как для разбиений $S(N, i)$ как минимум можно для первых $i-1$ слагаемых можно прибавить к ним единицу, тогда получим разбиения для числа $N+(i-1)$ на $k$ слагаемых. Также можно составить разбиение такое, что первые $i-2$ слагаемых увеличиваем на единицу, а $i-1$ слагаемое на $2$. Получили ещё одно дополнительное разбиение для числа $N+(i-1)$ на $k$ слагаемых, значит $S(N+(i-1), i) > S(N, i)$. 

Получаем такую цепочку неравенств:

\[i \cdot S(N+(i-1), i) > S(N+(i-1), i) > S(N, i)\]

Отсюда сразу получаем, что

\[\sum_{i=1}^k i \cdot S(N+(i-1), i) \text{ }>\text{ } \sum_{i=1}^k S(N,i)\]

Значит разбиений числа $N$ на не более
чем $k$ слагаемых меньше, чем разбиений числа $N + k$ на ровно $k$ слагаемых.\\


\begin{flushright}
\begin{large}
\textbf {Ответ: разбиений числа $N + k$ на ровно $k$ слагаемых больше}
\end{large}
\end{flushright}

{\bf 10.} Чего больше, правильных скобочных последовательностей из n
пар скобок или последовательностей ($x_1, x_2, \cdots , x_{2n}$) с элементами $\pm 1$, таких что $\sum\limits_{i = 1}^{2n} = 0$?
\begin{center}
\bfseries
{\Large Решение: }
\end{center}

Обозначив "$)$"\ за $+1$ , а "$($"\ за $-1$ получим, что любой последовательности из пар скобок можно поставить в соответствие точно такую же последовательность из $\pm 1$ (мы можем это сделать потому что кол-во ")" равно кол-ву "(" и тогда будет выполняться условие $\sum\limits_{i = 1}^{2n} = 0$ для последовательности из $\pm 1$), однако в обратную сторону это не сработает, т.к. на последовательность из $\pm 1$ ограничений в расположении нет,а на скобки есть( они должны располагаться правильными парами).

Таким образом каждому эл-ту из множества $X$ (множества скобок) соответствует один элемент из $Y$ , но существуют элементы из $Y$ , для которых не существует элемента из $X$. Иными словами мы получили инъекцию из $X$ в $Y$ . Значит последовательностей ($x_1, x_2, \cdots , x_{2n}$) с элементами $\pm 1$, таких что $\sum\limits_{i = 1}^{2n} = 0$ больше.

\begin{flushright}
\begin{large}
\textbf {Ответ: последовательностей ($x_1, x_2, \cdots , x_{2n}$) с элементами $\pm 1$, таких что $\sum\limits_{i = 1}^{2n} = 0$ -- больше.}
\end{large}
\end{flushright}

\end{document}


