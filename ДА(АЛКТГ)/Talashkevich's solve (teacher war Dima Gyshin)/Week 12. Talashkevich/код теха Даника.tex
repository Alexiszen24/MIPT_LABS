
\documentclass[a4paper,12pt]{article} % тип документа


% Русский язык
\usepackage[T2A]{fontenc} % кодировка
\usepackage[utf8]{inputenc} % кодировка исходного текста
\usepackage[english,russian]{babel} % локализация и переносы


% Математика
\usepackage{amsmath,amsfonts,amssymb,amsthm,mathtools}


\usepackage{wasysym}

%Заговолок
\author{Талашкевич Даниил Александрович}

\title{Неделя 12. Булевы функции}

\date{\today}

\begin{document}

\maketitle
\thispagestyle{empty}

\newpage
\setcounter{page}{1}
\begin{center}
\itshape
\bfseries
{ \Large Problems:}
\end{center}

{\bf 1.} Укажите существенные и несущественные (фиктивные) переменные
функции $f(x1, x2, x3) = 00111100$ и разложите её в ДНФ и КНФ.
\begin{center}
\bfseries
{\Large Решение: }
\end{center}
\[f(x_1,x_2,x_3) = 00111100\]

Разложим булеву функцию в ДНФ:
\[f(x_1,x_2,x_3) = (\overline{x_1} \wedge x_2 \wedge \overline{x_3})\vee (\overline{x_1} \wedge x_2 \wedge x_3) \vee\]
\[\vee (x_1 \wedge \overline{x_2} \wedge \overline{x_3}) \vee (x_1 \wedge \overline{x_2} \wedge x_3)\]

Заметим, что значения конъюнкции в первых двух скобках и в последних двух скобках не зависят от значения $x_3$ -- фиктивная переменная. Значит можно убрат её из рассмотрения:
\[f(x_1,x_2,x_3) = (\overline{x_1} \wedge x_2)\vee (x_1 \wedge \overline{x_2})\]

Теперь разложим эту функцию в КНФ:
\[f(x_1,x_2,x_3) = (\overline{x_1} \wedge x_2)\vee (x_1 \wedge \overline{x_2}) = \]
\[= ((\overline{x_1} \vee x_1) \wedge (x_1 \vee x_2))\wedge ((\overline{x_2} \vee x_2) \wedge (\overline{x_1} \vee \overline{x_2})) = (\overline{x_1} \vee \overline{x_2}) \wedge (x_1 \vee x_2) \]


\begin{flushright}
\begin{large}
\textbf {Ответы в решении}
\end{large}
\end{flushright}

{\bf 2.} Вычисляется ли константа $0$ в базисе $\{ \neg(x_1 \rightarrow x_2) \}$?
\begin{center}
\bfseries
{\Large Решение: }
\end{center}

Для того, чтобы константа $0$ вычислялась в базисе $\{ \neg(x_1 \rightarrow x_2) \}$ необходимо и достаточно, чтобы подобранная функция $f$ принимала $0$ для любых наборов $\{x_1,x_2\}$.

Рассмотрим значения базиса для разных наборов $x_1, x_2$:

\begin{center}
\begin{tabular}{|c|c|c|}
\hline 
$x_1$ & $x_2$ & $f(x_1,x_2)$ \\ 
\hline 
0 & 0 & 0 \\ 
\hline 
0 & 1 & 0 \\ 
\hline 
1 & 0 & 1 \\ 
\hline 
1 & 1 & 0 \\ 
\hline 
\end{tabular} 
\end{center}

Исходя из данных таблицы можем получить, каким образом вычисляется константа $0$ в исходном базисе $f(x_1,x_2)$. Для этого можем взять $f \rightarrow f$, тогда получим, что это выражение истинно для любых наборов $(x_1,x_2)$. Теперь, чтобы получить $0$ просто возьмем дизъюнкцию от этого выражения, что соответствует следующему выражению:

\[ f(f(x_1,x_2),f(x_1,x_2)) = 0  .\]

\begin{flushright}
\begin{large}
\textbf {Ответ: вычисляется.}
\end{large}
\end{flushright}

{\bf 3.} Вычислите $MAJ(x, y, z)$ схемой в базисе Жегалкина $\{1, \wedge, x_1 + x_2\}$.
\begin{center}
\bfseries
{\Large Решение: }
\end{center}
Полином Жегалкина функции от 3 переменных формально представим в виде:
\[f(x_1,x_2,x_3) = a_0 \oplus a_1x_1 \oplus a_2x_2 \oplus a_3x_3 \oplus a_{12}x_1x_2 \oplus a_{13}x_1x_3 \oplus a_{23}x_2x_3 \oplus a_{123}x_1x_2x_3\]

В формуле $7$ операций $\oplus$. Для различых значений аргументов имеем:
\[f(0,0,0) = a_0 \oplus 0 \oplus 0 \oplus 0 \oplus 0 \oplus 0 \oplus 0 \oplus 0 = a_0 \oplus 0 = 0 \Rightarrow a_0 = 0\]
\[f(0,0,1) = 0 \oplus 0 \oplus 0 \oplus a_3 \oplus 0 \oplus 0 \oplus 0 \oplus 0 = a_0 \oplus 0 = 0 \Rightarrow a_3 = 0\]

Из симметрии функции относительно переменных следует, что $a_1 = a_2 = a_3 = 0$.
\[f(x_1,x_2,x_3) = 0 \oplus a_{12}x_1x_2 \oplus a_{13}x_1x_3 \oplus a_{23}x_2x_3 \oplus a_{123}x_1x_2x_3\]
\[f(1,1,0) = 0 \oplus a_{12} \oplus 0 \oplus 0 \oplus 0 = 1 \Rightarrow a_{12} = 1\]
Из симметрии функции относительно переменных следует, что $a_{12} = a_{13} = a_{23} = 1$.
\[f(1,1,1) = 0 \oplus 1 \oplus 1 \oplus 1 \oplus a_{123} = 1 \Rightarrow a_{123} = 0\]
\[f(x_1,x_2,x_3) = 0 \oplus x_1x_2 \oplus x_1x_3 \oplus x_2x_3 \oplus 0 = 0 \oplus x_1x_2 \oplus x_1x_3 \oplus x_2x_3 = x_1x_2 \oplus x_1x_3 \oplus x_2x_3\]



\begin{flushright}
\begin{large}
\textbf {Ответ: $f(x_1,x_2,x_3) = x_1x_2 \oplus x_1x_3 \oplus x_2x_3$}
\end{large}
\end{flushright}

{\bf 4.} Сколько ненулевых коэффициентов в многочлене Жегалкина, который равен $x_1 \vee x_2 \vee ... \vee x_n$?
\begin{center}
\bfseries
{\Large Решение: }
\end{center}

Выразим многочлен в базисе Жегалкина:
$$x_1 \vee x_2 \vee ... \vee x_n = \neg(\neg(x_1 \vee x_2 \vee ... \vee x_n)) = \neg(\overline{x_1}\wedge \overline{x_2} \wedge ...\ \wedge \overline{x_n}) =$$
$$= \neg((1 + x_1)(1 + x_2)\cdot ... \cdot (1 + x_n)) = \neg(1 + A) = A.$$

Так как $1 + A$ содержит всего $2^n$ одночленов, то $A$ содержит $2^n - 1$ одночлен.
\begin{flushright}
\begin{large}
\textbf {Ответ: $2^n - 1$}
\end{large}
\end{flushright}

{\bf 5.} Докажите полноту базиса, состоящего из одной функции $x | y$,
которая по определению равна $\neg(x \wedge y)$ (штрих Шеффера, она же
NAND).
\begin{center}
\bfseries
{\Large Решение: }
\end{center}

Докажем полноту базиса от противного -- если бы это было не так, то этот базис относился бы к одному из следующих классов, согласно критерию Поста:

1) $T_0$: но функция может принимать значение $1$ при $x = y = 0$.

2) $T_1$: но функция может принимать значение $0$ при $x = y = 1$.

3) Проверим функцию на самодвойственность: $\overline{f(\overline{x},\overline{y})} = \overline{\neg(\overline{x} \wedge \overline{y})} = \overline{x} \wedge \overline{y} \neq \overline{x} \vee \overline{y}$ -- не самодвойственна

4) Линейность: штрих Шеффера представим через полином Жегалкина в виде $1 \oplus xy$, что уже противоречит линейности.

5) Монотонность: на значениях $x = y = 1$ функция принимает значение $0$ -- убывает, что противоречит монотонности.

\begin{flushright}
\begin{large}
\textbf {Доказано}
\end{large}
\end{flushright}

{\bf 6.} Является ли полным базис $\{ \wedge, \rightarrow \}$?
\begin{center}
\bfseries
{\Large Решение: }
\end{center}

Можно заметить, что набор функций замнут на классе $T_1$, так как выполняются следующие выражения:

{\bf 1)} $1 \wedge 1 = 1$;

{\bf 2)} $1 \rightarrow 1 = 1$.

Получили, что набор функций замкнут на классе $T_1$, значит этот набор не является полным базисом.

\begin{flushright}
\begin{large}
\textbf {Ответ: нет, не является.}
\end{large}
\end{flushright}

{\bf 7.} Является ли полным базис $\{\neg, MAJ(x_1, x_2, x_3)\}$?
\begin{center}
\bfseries
{\Large Решение: }
\end{center}

Рассмотрим функцию $MAJ$ на $3$ аргументах. Если $MAJ(x_1,x_2,x_3) = 1$, то $\neg MAJ(x_1,x_2,x_3) = 0$ и $MAJ(\overline{x_1},\overline{x_2},\overline{x_3}) = 0$, то есть $MAJ(\overline{x_1},\overline{x_2},\overline{x_3}) = \neg MAJ(x_1,x_2,x_3)$, полученный результат не зависит от того, какое значение принимало функция $MAJ(x_1,x_2,x_3)$. Отсюда следует, что функция $MAJ$ -- самодвойственная. Аналогично можно сказать про унарную функцию отрицания: $\neg x_1 = \neg ( \neg\overline{x_1})$ -- самодвойственная функция. Значит каждая функция базиса $(MAJ(x_1,x_2,x_3), \neg)$ самодвойственная и по критерию Поста базис не является полным.

\begin{flushright}
\begin{large}
\textbf {Ответ: не является.}
\end{large}
\end{flushright}

{\bf 8.} Пусть $f(x_1, ..., x_n)$ -- немонотонная функция. Докажите, что $\neg x_i $ вычисляется в базисе $\{0,1,f \}$.
\begin{center}
\bfseries
{\Large Решение: }
\end{center}

$[f \text{-- немонотонная функция} ]$ $\overset{\mathrm{def}}{=}$ $[\exists \text{ такой набор из } \{0,1\} \text{ для } \vec{A}, \text{ что }$

$f(\vec{A}) = 1] \wedge [\exists \vec{B} : f(\vec{B}) = 0]\wedge [B \text{ можно получить из } A \text{ заменой некото-}$

$\text{рого нуля на единицу}]$.

Последнее в свою очередь означает, что:

{\bf 1)} $\vec{A} = \{\alpha_1, \alpha_2, ... , \alpha_i,..., \alpha_n\},$ где $\alpha_i = 0$;

{\bf 2)} $\vec{B} = \{\beta_1, \beta_2, ... , \beta_i,..., \beta_n \},$ где $\beta_i = \overline{\alpha_i} = 1$.

Теперь, если $x_i = 1$, то $\neg x_i = 0$ и наоборот, тогда заметим, что $\neg x_i$ получается в базисе $\{0,1,f\}$ следующим образом:

\[\neg x_i = f(\vec{C}), \text{а } \vec{C} = \{\gamma_1,\gamma_2, ... , x_i, ... , \gamma_n\}.\]

Тогда, при $x_i = 1$ получаем, что $\vec{C}$ соответствует $\vec{B}$, а  в свою очередь $f(\vec{B}) = 0 = \neg x_i$.

Аналогично при $x_i = 0$ получаем, что  $\vec{C}$ соответствует $\vec{A}$, а  в свою очередь $f(\vec{A}) = 1 = \neg x_i$.


\begin{flushright}
\begin{large}
\textbf {Доказано}
\end{large}
\end{flushright}

{\bf 9.} Докажите, что всякую монотонную булеву функцию можно вычис-
лить монотонной схемой (с базисом $\wedge, \vee, 1, 0)$.
\begin{center}
\bfseries
{\Large Решение: }
\end{center}

Нужно доказать, что всякая монотонная булева функция выражается в ДНФ без связки $\neg$, так как с этой связкой базис являлся бы полным. Пусть случайная монотонная булева функция принимает $n$ аргументов.

Рассмотрим такой набор переменных, что значение функции  равно $1$, при этом количество аргументов, равных $1$, минимально. Так как функция монотонная, то существуют
другие наборы переменных, значение функции на которых равно $1$, которые отличаются от рассматриваемой только тем, что некоторая переменная, принимающая $0$, теперь принимает $1$. В ДНФ эти два набора будут иметь вид $... \vee (\neg a_i \wedge a_{i+1} \wedge...) \vee (a_i \wedge a_{i+1}...) \vee ...$

Эта дизъюнкция двух наборов равна $(a_{i+1}\wedge...)$, то есть перерь не зависит от $a$.  Таким образом, любую дизъюнкцию конъюнкций с отрицанием можно сократить
до конъюнкции без отрицания, значит, ДНФ представление любой монотонной функции можно сократить до представления ДНФ без отрицания. что и требовалось доказать.


\begin{flushright}
\begin{large}
\textbf {Доказано}
\end{large}
\end{flushright}

{\bf 10.} Булева функция $PAR(x_1,x_2,...,x_n)$ равна 1, если количество единиц среди значений $x_1,x_2,...,x_n$ нечётно и нулю, если чётно.
\begin{center}
\bfseries
{\Large Решение: }
\end{center}

{\bf 1)} Вполне очевидно, что $PAR(x_1,x_2,\ ...\ ,x_n) = x_1 + x_2 + ... + x_n$. И вправду, ведь $PAR(x_1,x_2,\ ...\ ,x_n) = x_1 + x_2 + ... + x_n = k\cdot 1 + 0 \cdot (n - k) = k\cdot 1 = 1$, для $k\ - $ нечетного.

{\bf 2)} Докажем, что фукнция задается в виде ДНФ без отрицаний только если она монотонна.

В действительности, любую функцию можно представить в виде $f = x_n \varphi \vee \overline{x_n}\cdot \psi$. Действительно, в зависимости от значения $x_n$ мы имеем значение функции $f = \varphi$ или $f = \psi$, а они уже задаются остальными $n - 1$ переменными. Рассмотрим некоторые случаи и покажем зависимости между $\phi$ и $\varphi$.

$\bullet$ $x_n = 0 : f = \psi,$

$\bullet\ x_n = 1: f = \varphi.$

Наша функция $f$ должна быть монотонной, так что 

$\bullet\ \varphi = 1, \psi = 0,$

$\bullet\ \varphi = \psi.$

Теперь докажем, что $x_n \cdot \varphi \vee \overline{x_n} \cdot \psi = \varphi \cdot x_n \vee \psi$. Действительно, если $\varphi = 1$, $\psi = 0$, то очевидно, что функции эквивалентны, так как второе слагаемое больше ничего не решает.

В случае, если $\varphi = \psi$, то по закону поглощения функции так же равны.

Таким образом, при <расширении> функции на одну переменную мы показали, что ДНФ можно представить и без отрицания вовсе. Для одной переменной это очевидно, а далее по аксиоме индукции, с учетом доказательства, указанного выше, мы имеем, что монотонной функция является тогда и только тогда, когда представима в ДНФ без отрицаний, что и требовалось доказать.


Теперь, так как функция задается в виде ДНФ без отрицаний только если она монотонна, получаем, что при $n > 1$ $PAR$ -- немонотонна (она немонотонна, так как $\exists$ набор такой, что $PAR(1,0,0,\ ...\ ,0) = 1$ и $PAR(1,1,0,\ ...\ ,0) = 0$ -- значит $PAR$ немонотонна) $\Rightarrow n \leqslant 1,$ а так как $n \geqslant 1 \Rightarrow$ единственное возможное значение $n = 1$. 

\begin{flushright}
\begin{large}
\textbf {Ответ: $1)\ PAR(x_1,x_2,\ ...\ ,x_n) = x_1 + x_2 + ... + x_n.$\\
$2) n = 1.$}
\end{large}
\end{flushright}

{\bf 11.} Булева функция $f : \{0, 1\}^n \rightarrow \{0, 1\}$ называется линейной, если она представляется в виде
\[ f(x_1,...,x_n) = a_0 + (a_1 \wedge x_1) + \dots +(a_n \wedge x_n). \]

для некоторого набора $(a_1, . . . , a_n) \in \{0, 1\}^n$ булевых коэффициентов.

Докажите, если $f(x_1, . . . , x_n)$ -- нелинейная функция, то конъюнк-
ция $x_1 \wedge x_2$ вычисляется схемой в базисе $\{0, 1, \neg, f\}$.
\begin{center}
\bfseries
{\Large Решение: }
\end{center}

Представим нелинейную функция в виде полинома Жегалкина: 
\[f(x_1,x_2,...,x_n) = a_0 \oplus a_1x_1 \oplus a_2x_2 \oplus ...\oplus a_{12}x_1x_2 \oplus...\]

Полином будет как содержать как минимум одно слагаемое-конъюнкцию вида $x_i x_j...$. Возьмём это слагаемое. Тогда на $i$-ую и $j$-ую позиции поставим $x_i$ и $x_j$ -- аргументы требуемой конъюнкции, и рассмотрим значение функции:
\[f(0,0,...,x_i,...,0,0,...,x_j,...,0,0,...) = 0 \oplus 0 \oplus ... \oplus x_ix_j \oplus ... \oplus 0 \oplus 0 \oplus ...\]

Учитывая, что $0 \oplus 0 = 0$, получим:
\[f(0,0,...,x_i,...,0,0,...,x_j,...,0,0,...) = 0 \oplus x_ix_j\]

Если $x_ix_j = 1$, то $f(0,0,...,x_i,...,0,0,...,x_j,...,0,0,...) = 1$, если $x_ix_j = 0$, то $f(0,0,...,x_i,...,0,0,...,x_j,...,0,0,...) = 0$, значит 
\[f(0,0,...,x_i,...,0,0,...,x_j,...,0,0,...) = x_ix_j\]

Значит на $i$-ую и $j$-ую позиции поставим $x_1$ и $x_2$:
\[x_1 \wedge x_2 = f(0,0,...,x_1,...,0,0,...,x_2,...,0,0,...)\]

\begin{flushright}
\begin{large}
\textbf {Доказано}
\end{large}
\end{flushright}

{\bf 12.} Докажите теорему Поста.
\begin{center}
\bfseries
{\Large Решение: }
\end{center}

Теорема Поста: $\{f_1 ... f_n \}$ полная $\Leftrightarrow$ среди $f_1 ... f_n$ найдется по функции, которые не принадлежат каждому из $5$ классов (т.~е. хотя бы одна из них не лежит в $T_0$, хотя бы одна $\notin T_1$ ... хотя бы одна $\notin M$ ). 

Доказательство:

($\Rightarrow$) если все $f_1 ... f_n$ лежат в одной из этих классов,то замыкание не может быть больше этого класса $\Rightarrow$ система не полна.

($\Leftarrow$) Первый шаг: получим $0,1, \neg$ 

Пусть $f_0 \notin T_0, f_1 \notin T_1, g \notin S, h \notin M, k \notin L$.

$\bullet$ $f_0 (0...0) = 1$. Если $f_0(1...1) = 0$, то $f(p...p) = \neg p$. Если же $f(1...1) = 1$, $f(p...p) = 1$.

$\bullet$ $f_1 (1...1) = 0$. Если $f_1(0...0) = 0$, то $f_1(p...p) = 0$, иначе $f_1(p...p) = \neg p$.

Три случая: 

{\bf 1)} В одном случае $\neg$, в другом константа $\Rightarrow$ получаем другую константу (и $g$ и $h$ даже не нужны)

{\bf 2)} Если получили $0$ и $1$, то берем $h$: есть набор, на котором $h(a_1... a_{i - 1};$

$0;a_{i+ 1}... a_n)  = 1$ и $h(a_1... a_{i - 1};1;a_{i+ 1}... a_n)  = 0$ $\Rightarrow$

$\Rightarrow \neg p = h(a_1...a_{i - 1};p;a_{i + 1}...a_n).$

{\bf 3)} Если получена $\neg$ в обоиэ случаях : $g \notin S \Rightarrow \exists \vec{a}: g(\vec{a}) = g (\overline{\vec{{a}}})$.

Первый шаг закончен.

Второй шаг: берем $0,1,\neg$ и функция $K$. У функции $K$ многочлен Жегалкина содержит нелинейность. Без ограничения общности, есть произведение $\alpha \cdot x_1\cdot x_2$, где $\alpha$ -- какой-то коэффициент, а $x_1,x_2$ -- разложение самого многочлена.

Представим $K(x_1...x_n) = x_1x_2\cdot A(x_3...x_n) + x_1B(x_3...x_n) + x_2C(x_3...x_n) + D(x_1...x_n)$, причем $A(x_3...x_n) \neq 0.$

Берем набор $\alpha_3...\alpha_n$ такой, что $A(\alpha_3...\alpha_n) = 1 \Rightarrow$ подставим и получим $x_1x_2 + bx_1 + cx_2 + d$. Взяв от этого отрицание, если $\alpha = 1$, сведем к $x_1x_2 + bx_1 + cx_2$.

Если $b = c =0$, то имеем $x_1\wedge x_2$ и $\neg$.

Если $b = c =1$, то $x_1x_2 + x_1 + x_2 = x_1 \vee x_2$, а также есть $\neg$.

Если $b = 1, c = 0$, то получим $x_1x_2 + x_1 = x_1(x_2 + 1) = x_1\overline{x_2} = \overline{x_1\rightarrow x_2} \Rightarrow$ получим суперпозиции $\overline{x_1 \rightarrow \overline{x_2}} = x_1 \wedge x_2.$

\begin{flushright}
\begin{large}
\textbf {Доказано}
\end{large}
\end{flushright}

\end{document}


