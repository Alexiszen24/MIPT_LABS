\documentclass[a4paper, 12pt]{article} % тип документа

%%%Библиотеки
	%\usepackage[warn]{mathtext}	
	\usepackage[english, russian]{babel} %Локализация и переносы
	\usepackage{caption}
	\usepackage{listings}
	\usepackage{amsmath, amsfonts, amssymb, amsthm, mathtools}
	\usepackage[warn]{mathtext}
	\usepackage[mathscr]{eucal}
	\usepackage{wasysym}
	\usepackage{graphicx} %Вставка картинок правильная
	\usepackage{indentfirst}
	\usepackage{float}    %Плавающие картинки
	\usepackage{wrapfig}  %Обтекание фигур (таблиц, картинок и прочего)
	\usepackage{fancyhdr} %Загрузим пакет
	\usepackage{lscape}
	\usepackage{xcolor}
	\usepackage[normalem]{ulem}
	
	\usepackage{titlesec}
	\titlelabel{\thetitle.\quad}

	\usepackage{hyperref}

%%%Конец библиотек

%%%Настройка ссылок
	\hypersetup
	{
		colorlinks = true,
		linkcolor  = blue,
		filecolor  = magenta,
		urlcolor   = blue
	}
%%%Конец настройки ссылок


%%%Настройка колонтитулы
	\pagestyle{fancy}
	\fancyhead{}
	\fancyhead[L]{2.2.1}
	\fancyhead[R]{Талашкевич Даниил, группа Б01-009}
	\fancyfoot[C]{\thepage}
%%%конец настройки колонтитулы



\begin{document}
						%%%%Начало документа%%%%


%%%Начало титульника
\begin{titlepage}

	\newpage
	\begin{center}
		\normalsize Московский физико-технический институт \\(госудраственный университет)
	\end{center}

	\vspace{6em}

	\begin{center}
		\Large Лабораторная работа по общему курсу физики\\Термодинамика и молекулярная физика
	\end{center}

	\vspace{1em}

	\begin{center}
		\Large \textbf{2.2.1. Исследование диффузии газов}
	\end{center}

	\vspace{2em}

	\begin{center}
		\large Талашкевич Даниил Александрович\\
		Группа Б01-009
	\end{center}

	\vspace{\fill}

	\begin{center}
		Долгопрудный \\19.04.2021
	\end{center}
	
\end{titlepage}
%%%Конец Титульника


					%%%%%%Начало работы с текстом%%%%%%

Пусть имеются 2 разных газа, например воздух и этилен. Концентрацию этилена считаем малой по сравнению с концентрацией воздуха. Изучим их взаимную диффузию. Пусть $\frac{dN_{12}}{dt}$ -- количество столкновений молекулы газа 1 с молекулами газа 2  в единицу времени, аналогично введём $\frac{dN_{12}}{dt}$. Пусть $\Delta p_1$ -- усреднённое изменение импульса молекулы газа 1 за одно столкновение. Значит в единицу времени полное измнение импульса молекулы газа 1 равно $\Delta p_1 \frac{dN_{12}}{dt} = F_1$ -- средняя сила, действующая на молекулу  газа 1.  

Найдём величину $\frac{dN_{12}}{dt}$. Эта величина определяется стандратным выражением $\frac{dN_{12}}{dt} = n_2 \sigma_{12} v_{\text{отн}}$, которое можно получить, расписывания число соударений через соударения с молекулами газа 1 через раписывания числа соударений через концентрацию и т.д.. Так как температура газов  одинакова, то в среднем $m_1 v_1^2 = m_2 v_2^2$, а
\begin{equation}
	v_{\text{отн}} = v_1 \sqrt{\frac{m_1}{\mu}} = v_1 \sqrt{\frac{m_1 + m_2}{m_2}}
\end{equation}

\begin{equation}
	\frac{dN_{12}}{dt} = n_2 \sigma_{12} v_{\text{отн}} = n_2 \sigma_{12} v_1 \sqrt{\frac{m_1 + m_2}{m_2}} \Rightarrow \lambda = \frac{1}{n_2 \sigma_{12} \sqrt{\frac{m_1 + m_2}{m_2}}}
\end{equation}

\begin{equation}
	v_1 = \sqrt{\frac{8RT}{\pi M}}
\end{equation}

\begin{equation}
	D = \frac{1}{3} \sqrt{\frac{8RT}{\pi M}} \frac{1}{n_2 \sigma_{12} \sqrt{\frac{m_1 + m_2}{m_2}}}
\end{equation}

В нашей ситуации $M = 28$ г/моль. Эта величина почти равна молярной массе воздуха, поэтому 
\begin{equation}
	D = \frac{1}{3} \sqrt{\frac{8RT}{\pi M}} \frac{1}{n_2 \sigma_{12} \sqrt{2}} = \frac{2}{3} \sqrt{\frac{RT}{\pi M}} \frac{kT}{\sigma_{12} p} =
\end{equation}

\[ = \frac{8}{3} \sqrt{\frac{RT}{\pi M}} \frac{kT}{\pi (d_1 + d_2)^2 p} = 9 \cdot 10^{-6} m^2/c\]




\end{document}