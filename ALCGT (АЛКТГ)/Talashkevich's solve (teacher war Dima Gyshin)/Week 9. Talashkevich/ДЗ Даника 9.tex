
\documentclass[a4paper,12pt]{article} % тип документа


% Русский язык
\usepackage[T2A]{fontenc} % кодировка
\usepackage[utf8]{inputenc} % кодировка исходного текста
\usepackage[english,russian]{babel} % локализация и переносы


% Математика
\usepackage{amsmath,amsfonts,amssymb,amsthm,mathtools}


\usepackage{wasysym}

%Заговолок
\author{Талашкевич Даниил Александрович}

\title{Неделя 9. Комбинаторика III. 
Формула включений-исключений}

\date{\today}

\begin{document}

\maketitle
\thispagestyle{empty}

\newpage
\setcounter{page}{1}
\begin{center}
\itshape
\bfseries
{ \Large Problems:}
\end{center}

{\bf 1.} Сколькими способами можно закрасить клетки таблицы 3 × 4 так,
чтобы незакрашенные клетки содержали или верхний ряд, или нижний
ряд, или две средних вертикали?
\begin{center}
\bfseries
{\Large Решение: }
\end{center}

Обозначим за $A$ -- множество способов, где не закрашен верхний ряд, $B$ -- нижний ряд, а $C$ -- две средних вертикали. Тогда по формуле включений исключений имеем:
\[S = |A| + |B| + |C| - |A \cap B| - |A \cap C| - |C \cap B| + |A \cap B \cap C| = \]
\[= 2^8 + 2^8 + 2^6 - 2^4 - 2^4 - 2^4 + 2^2 = 532\]


\begin{flushright}
\begin{large}
\textbf {Ответ:  532 способа.}
\end{large}
\end{flushright}

{\bf 2.} Для полета на Марс набирают группу людей, в которой каждый
должен владеть хотя бы одной из профессий повара, медика, пилота
или астронома. При этом в техническом задании указано, что каждой
профессией из списка должно владеть ровно $6$ человек в группе. Кроме

того указано, что в группе должен найтись ровно один человек, вла-
деющий всеми этими профессиями; каждой парой профессий должны

владеть ровно $4$ человека; каждой тройкой -- ровно $2$.
Выполнимо ли такое техническое задание?
\begin{center}
\bfseries
{\Large Решение: }
\end{center}

Подсчитаем общее количество человек в группе через формулу включений и исключений. Всего есть $4$ профессии, и каждой процессией должно владеть $6$ человек. Поскольку каждый владеет хотя бы одной профессией, общее число равно $6\cdot 4 - 4\cdot C_4^2 + 2\cdot C_4^3 - 1$ = $7$. Здесь учтено, что двойных пересечений $C_4^2 = 6$, а тройных $C_4^3 = 4$. Тогда пусть $6$ человек владеют первой профессией, как минимум другой профессией будут владеть $6 - (7 - 6) = 6 - 1 = 5$ человек, а по условию двумя процессиями могут владеть только $4$ человека.



\begin{flushright}
\begin{large}
\textbf {Ответ: задание не выполнимо}
\end{large}
\end{flushright}

{\bf 3.} Пусть $A$ и $B$ -- конечные непустые множества, и $|A| = n$. Известно, что число инъекций из $A$ в $B$ совпадает с числом сюръекций из $A$ в $B$. Чему равно это число?
\begin{center}
\bfseries
{\Large Решение: }
\end{center}

Пусть $|B| = m$. Если можно построить инъекцию из $A$ в $B$, то $n \leqslant m$. Аналогично, если существует сюръекция из $A$ в $B$, то $m \leqslant n$. Значит $m = n$. Число инъекций из $A$ в $B$ равно
\[N = \frac{n!}{(n- m)!} = \frac{n!}{(n-n)!} = n!\]


\begin{flushright}
\begin{large}
\textbf {Ответ: $n!$}
\end{large}
\end{flushright}

{\bf 4.} Пусть $X = \{1,\dots ,n \}$. Найдите число способов взять $k$ подмножеств $X_1,X_2,\dots ,X_k$ множества $X$ таких, что $X_1 \subseteq X_2 \subseteq \dots \subseteq X_k$.
\begin{center}
\bfseries
{\Large Решение: }
\end{center}

Рассмотрим упорядоченную последовательность из $n$ символов (соответствуют элементам множества $X$): каждый элемент может принимать значение от $0$ до $k$. Элемент последовотельности отвечает за то, в какое подножество он входит: равен $0$, если ни в одно не входит, или принимает количество подножеств, в которые входит. 

Заметим, что любая такая последовательность явно задаёт единственную комбинацию подножеств $X_1, X_2, ..., X_k$: элемент, входящий в подножества наибольшее количество раз $m$ расставим в последние $m$ подножеств. Если элемент входит $l < m$ раз, то расставим этот элемент в последние $l$ подножеств. Таким образом некоторые подножества могут быть пустыми. Если же некоторые два подножества равны, то мы можем их переставить и ничего не изменится -- считаем за $1$ способ.

Значит имеем $(k+1)^n$ способов выбрать подножества.


\begin{flushright}
\begin{large}
\textbf {Ответ: :$(k+1)^n$ способов}
\end{large}
\end{flushright}

{\bf 5.} В классе $20$ учеников, каждый из которых дружит ровно с шестью
одноклассниками. Найдите число таких различных компаний из трёх
учеников, что в них либо все школьники дружат друг с другом, либо
каждый не дружит ни с одним из двух оставшихся.
\begin{center}
\bfseries
{\Large Решение: }
\end{center}

Для школьника существует всего $3$ ситуации: либо он дружит с обоими из двух случайно выбранных школьников, либо только с одним. либо ни с кем. Выбрать оного ученика -- $20$ способов, выбрать его друга -- $6$ способов, не его друга -- $13$ способов. Учитывая, что такими комбинациями мы учли каждый способ выбора $2$ раза, получаем, что есть $\frac{20 \cdot 13 \cdot 6}{2} = 780$ способов выбрать $3$ школьника, не удовлетворяющих условиями.

Способов выбрать $3$ школьника из $20$ равно $C^3_{20} = 1140$. Значит всего существует $1140-780 = 360$ способов выбрать таких $3$ школьника, которые удовлетворяют условяим.


\begin{flushright}
\begin{large}
\textbf {Ответ:  $360$ компаний}
\end{large}
\end{flushright}

{\bf 6.} Найдите количество неубывающих отображений \[f : \{ 1,2,\dots , n\} \rightarrow \{ 1,2,\dots ,m\}. \] 
\begin{center}
\bfseries
{\Large Решение: }
\end{center}

Расставим перегородки между значениями функции. Значению между ($i-1)$-й и $i$-й перегородкой соответствует $x_i$, если между перегородками ничего нет, то $x_i$ соответствует предыдущему значению. Так как у нас $n$ аргументов и $m$ значений, то нам нужно выбрать $m-1$ мест под
перегородки из $m + n-1$ возможных. Что дает $C_{n+m-1}^{m-1}$.

\begin{flushright}
\begin{large}
\textbf {Ответ: $C_{n+m-1}^{m-1}$.}
\end{large}
\end{flushright}


{\bf 7.} Чего больше, разбиений $n$-элементного множества на не более чем $k$ подмножеств или разбиений $(n + k)$-элементного множества на ровно $k$
подмножеств? Определение разбиения приведено в классной работе $7$.
\begin{center}
\bfseries
{\Large Решение: }
\end{center}

разом построим инъекцию из множества разбиений $n$-элементного множества на не более чем $k$ подмножеств во множество разбиений ($n+k$)-элементного множества на ровно $k$ подмножеств.

Рассмотрим разбиения $n$-элементного множества на $m \leqslant k$ частей. Пронумеруем эти части произвольным образом. У нас есть $k$ дополнительных элементов. Поместим первый из них в первую часть, второй во вторую, и так далее до тех пора, пока не закончатся части. Из оставшихся дополнительных элементов сформируем одноэлементные части. Получится разбиение $n+k$ элементов на $k$ частей. Из этого разбиения строим прообраз, убрав все вставленные дополнительные элементы.

Так как при $k\geqslant 2$ у нас не менее двух частей разбиения, то мы можем эти части перенумеровать другим образом так, чтобы при новом построении разбиения $n+k$ элементов на $k$ частей выше описанным способом получилось разбиение, отличное от ранее построенного. В этом случае мы построили инъекцию из множества разбиений $n$-элементного множества на не более чем $k$ подмножеств во множество разбиений ($n+k$)-элементного множества на ровно $k$ подмножеств.

Если же $k=1$, то очевидно, что количество разбиений $n$-элементного множества на не более чем $1$ подмножество меньше количества разбиений ($n+1$)-элементного множества на ровно $1$ подмножеств.


\begin{flushright}
\begin{large}
\textbf {Ответ: доказано}
\end{large}
\end{flushright}

{\bf 8.} Сколькими способами можно рассадить за круглым столом n пар
влюблённых так, чтобы ни одна пара влюблённых не сидела рядом.
\begin{center}
\bfseries
{\Large Решение: }
\end{center}

Нам нужно рассадить ровно $k$ пар влюбленных рядом.

Рассмотрим первую пару: посадить первого можно на $2n$ мест и на одно из $2$ мест второго. Так как месте пронумерованы, то от их перестановки мы уже учли, значит посадить первую пару мы можем $4n$ способами. 

Остальные пары (их $k - 1$)мы будем рассматривать как единое целое,а остальных людей $2n-2k$  мы можем переставлять с $(k - 1)$ парами способами $(2n - k - 1)!$.

Так как в каждой паре у нас можно поменяться местами влюбленных, то еще получаем $2^{k-1}$ способов. Учитывая все выше получаем в итоге 
\[ 4n\cdot (2n - k - 1)!\cdot 2^{k-1} = 2n(2n - k - 1)! 2^k.\]

Каждую $k$-ую пару можно выбрать $C_n^k$ способом.

Используя формулу включений,исключений, получаем 
\[ A = \sum\limits_{k = 0}^{n}(-1)^k\cdot 2n\cdot 2^k\cdot (2n - k -1)!\cdot C_n^k.\] 
 
\begin{flushright}
\begin{large}
\textbf {Ответ: $ A = \sum\limits_{k = 0}^{n}(-1)^k\cdot 2n\cdot 2^k\cdot (2n - k -1)!\cdot C_n^k.$ }
\end{large}
\end{flushright}


{\bf 9.} Есть $n$ конфет и $m$ коробок. Найдите число способов разместить конфеты по коробкам для каждого из условий (все конфеты должны быть разложены): {\bf а)} и конфеты и коробки разные; {\bf б)} конфеты одинаковые, коробки разные, не должно быть пустых коробок; {\bf в)} конфеты одинаковые, коробки разные; {\bf г)} и конфеты и коробки разные, не должно быть пустых коробок; {\bf д)} конфеты разные, коробки одинаковые, не должно быть пустых коробок; {\bf е)} конфеты разные, коробки одинаковые.
\begin{center}
\bfseries
{\Large Решение: }
\end{center}

а) Есть $n$ способов выбрать конфету, которую можно положить в коробку, выбрать которую можно $m$ способами. Значит имеем $n^m$ способов.

б) Представим, что во множестве конфет нужно расставить $m-1$ перегородку так, чтобы вышло $m$ непустых коробок с конфетами. Количество способов выбрать $m-1$ перегородок из $n-1$ возможных равно $C^{m-1}_{n-1}$

в) Снова воспользуемся аналогией с перегородками, но в этот раз коробки могут быть пустые -- задаче эквивалентна задаче Муавра, значит есть $C^{m-1}_{n + m-1}$ способов.

г) Пусть конфеты -- аргументы некоторой функции, а коробки -- её значения. Так как коробки непустые, нужно найти количество сюръекций из множества конфет во множество коробок. Значит ответом является
\[\sum_{k=0}^{m-1} (-1)^k C^k_m (m-k)^n\]

д) Та же идея с сюръекциями, что и в предыдущем пункте, только нужно учесть, коробки одинаковые, значит нужно поделить ответ на количество перестановок коробок:
\[\frac{\sum_{k=0}^{m-1} (-1)^k C^k_m (m-k)^n}{m!}\]

е) Идея та же, что и в пункте а), только нужно учесть перестановки коробок, тогда ответ равен
\[\frac{m^n}{m!}\]


\begin{flushright}
\begin{large}
\textbf {Ответ: Ответы в решении}
\end{large}
\end{flushright}

{\bf 10.} Докажите справедливость равенства с помощью метода характе-
ристических функций: \[ | A_1\bigtriangleup \dots \bigtriangleup A_n | = \sum\limits_n |A_i| - 2 \sum\limits_{i < j}|A_i \cap A_j| + 4 \sum\limits_{i<j<k}|A_i\cap A_j \cap A_k| - \dots \]
\begin{center}
\bfseries
{\Large Решение: }
\end{center}

Найдём, чему равна характеристическая функция множества $A\bigtriangleup B$, где $A$, $B$ -- какие-то множества.
\[\chi_{A\bigtriangleup B} = \chi_{(A \cup B) \setminus (A \cap B)} = \chi_{A\cup B}(1 - \chi_{A\cap B}) =\]
\[ = (\chi_A + \chi_B - \chi_A \cdot \chi_B )(1 - \chi_A \cdot \chi_B) = \chi_A + \chi_B - 2\chi_A\chi_B\]

Теперь проделаем то же самое, добавив множество $C$:
\[\chi_{A\bigtriangleup B \bigtriangleup C} = \chi_{A\bigtriangleup B} + \chi_C - 2\chi_{A\bigtriangleup B}\chi_C =\]
\[ = \chi_A + \chi_B + \chi_C - 2\chi_A\chi_B - 2\chi_A\chi_C - 2\chi_B\chi_C + 4\chi_A\chi_B\chi_C\]

Из двух выражений можно сделать предположение, что
\[\chi_{A_1 \bigtriangleup A_2 \bigtriangleup ... \bigtriangleup A_n} = \sum_{i=1}^n \chi_{A_i} + (-2)^1\sum_{i < j}^n \chi_{A_i}\chi_{A_j} + (-2)^2\sum_{i < j < p}^n \chi_{A_i}\chi_{A_j}\chi_{A_p} + ...\]

С помощью индукции проверим, выполняется ли эта формула. Если $n=1$, то равенство очевидно. Пусть $n=k$ и выполняется равенство. Тогда для $n=k+1$ имеем:
\[\chi_{A_1 \bigtriangleup A_2 \bigtriangleup ... \bigtriangleup A_{k+1}} =\chi_{A_1 \bigtriangleup A_2 \bigtriangleup ... \bigtriangleup A_k} + \chi_{A_{k+1}} - 2\chi_{A_1 \bigtriangleup A_2 \bigtriangleup ... \bigtriangleup A_k} \chi_{A_{k+1}} =\]
\[= \sum_{i=1}^{k+1} \chi_{A_i} + \Big((-2)^1\sum_{i < j}^k \chi_{A_i}\chi_{A_j} + (-2)^2\sum_{i < j < p}^k \chi_{A_i}\chi_{A_j}\chi_{A_p} + ...\Big) - \]
\[-2 \chi_{A_{k+1}}\Big(\sum_{i=1}^{k} \chi_{A_i}+ (-2)^1\sum_{i < j}^k \chi_{A_i}\chi_{A_j} + (-2)^2\sum_{i < j < p}^k \chi_{A_i}\chi_{A_j}\chi_{A_p} + ...\Big) =\]
\[= \sum_{i=1}^{k+1} \chi_{A_i} +(-2)^1\Big(\sum_{i < j}^k \chi_{A_i}\chi_{A_j} + \sum_{i}^k \chi_{A_i}\chi_{A_k+1}\Big) +\]
\[+ (-2)^2\Big( \sum_{i < j < p}^k \chi_{A_i}\chi_{A_j}\chi_{A_p} + \sum_{i < j}^k \chi_{A_i}\chi_{A_j}\chi_{A_{k+1}}\Big) + ... = \]
\[= \sum_{i=1}^{k+1} \chi_{A_i} + (-2)^1\sum_{i < j}^{k+1} \chi_{A_i}\chi_{A_j} + (-2)^2\sum_{i < j < p}^{k+1} \chi_{A_i}\chi_{A_j}\chi_{A_p} + ...\]

Значит предположение было верно и равенство выполняется. Перейдём от характеристических функций с мощностью множеств:
\[|A_1 \bigtriangleup A_2 \bigtriangleup ... \bigtriangleup A_n| = \sum_{i=1}^{k+1} |A_i| + (-2)^1\sum_{i < j}^{k+1} |A_i\cap A_j| + (-2)^2\sum_{i < j < p}^{k+1} |A_i\cap A_j\cap A_p| + ...\]


\begin{flushright}
\begin{large}
\textbf {Ответ: Доказано}
\end{large}
\end{flushright}

\end{document}


