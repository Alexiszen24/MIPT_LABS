
\documentclass[a4paper,12pt]{article} % тип документа


% Русский язык
\usepackage[T2A]{fontenc} % кодировка
\usepackage[utf8]{inputenc} % кодировка исходного текста
\usepackage[english,russian]{babel} % локализация и переносы


% Математика
\usepackage{amsmath,amsfonts,amssymb,amsthm,mathtools}


\usepackage{wasysym}

%Заговолок
\author{Талашкевич Даниил Александрович}

\title{Неделя 5.Графы II. Деревья и раскраски}

\date{\today}

\begin{document}

\maketitle
\thispagestyle{empty}

\newpage
\setcounter{page}{1}
\begin{center}
\itshape
\bfseries
{ \Large Problems:}
\end{center}

{\bf 1.} Степень каждой вершины графа равна 2. Верно ли, что этот граф
2-раскрашиваемый?
\begin{center}
\bfseries
{\Large Решение: }
\end{center}

По теореме "граф $G$ является двураскрашиваемым тогда и только тогда, когда в
нём нет циклов нечётной длины". Значит наличие у графа степеней всех вершин равных 2 не является достаточным условием, потому что может содержаться цикл нечётной длины. Пример наличия цикла нечетной длины -- граф-треугольник.

\begin{flushright}
\begin{large}
\textbf {нет }
\end{large}
\end{flushright}

{\bf 2.} Докажите, что в дереве на $2n$ вершинах есть независимое множество
размера $n$ (ни одна пара вершин множества не соединена ребром). 
\begin{center}
\bfseries
{\Large Решение: }
\end{center}

Так как любое дерево 2-раскрашиваемо (существует правильная раскраска в 2 цвета), то количество вершин какого-то одного цвета будет не меньше, чем $n$, иначе получим, что $(x,y<n),$ а $ x+y < 2n$, а кол-во вершин должно равняться $2n$. То же самое по принципу Дирихле. Тогда возьмем $n$ вершин цвета, который встречается раз не меньше, чем $n$, все эти вершины будут раскрашены в один цвет и, соответственно, не будут смежными, тогда они будут составлять независимое множество размера $n$.

То, что любое дерево 2-раскрашиваемо (существует правильная раскраска в 2 цвета) легко можно доказать:

Вершина $v $ связана с каждой вершиной $u$ дерева $T$ единственным простым путем
$p_u$. Если длина $|p_u|$ пути $p_u$ четна, окрасим $u$ в белый цвет, иначе — в черный. Пусть
вершины $u$ и $w$ смежны. Если w не входит в путь $p_u$, то $p_w = p_uw$; в противном
случае $|p_w| < |p_u|$, а значит, $u$ не входит в $p_w$ и $p_u = p_wu$. В каждом из случае
числа $|p_u|$ и $|p_w|$ разной четности. Следовательно, вершины $ u$ и $w$ разного цвета и раскраска правильная.\\



\begin{flushright}
\begin{large}
\textbf {доказано}
\end{large}
\end{flushright}
\newpage
{\bf 3.} В дереве на $2020$ вершинах ровно три вершины имеют степень $1$.
Сколько вершин имеют степень $3$? 
\begin{center}
\bfseries
{\Large Решение: }
\end{center}

Очевидно, что вершины со степенью $degA \geqslant 4$ не может быть:

Если бы такая вершина существовала, то из такой вершины как минимум существовало бы 4 пути таких, что они заканчиваются на вершинах со степенью 1, иначе появился бы цикл. Значит было бы как минимум $degA$ вершин со степенью 1, что противоречит условиям. Значит $degA_{max} = 3$.

Так как из вершины $A$ в каждую вершину со степенью 1 ведет ровно 1 путь, то такая вершина обязательно должна существовать и она единственна.

\begin{flushright}
\begin{large}
\textbf {одна вершина со степенью 3 }
\end{large}
\end{flushright}

{\bf 4. }Есть два дерева на $ n$ вершинах, каждое имеет диаметр длины $d$.
Можно ли так добавить ребро между вершинами этих деревьев, чтобы
длина диаметра полученного дерева равнялась $d$? 
\begin{center}
\bfseries
{\Large Решение: }
\end{center}

Стоит сразу понять, что диаметр будет не меньше, чем исходный диаметр $d$, так как новое ребро никаким образом не уменьшит максимальное из расстояний между двумя вершинами исходных деревьев.

По принципу Дирихле наименьший возможный диаметр $d^{\textbf{'}}$ будет, если разместить ребро между центрами деревьев, то есть между их вершинами $\frac{n}{2}$. Если ребро соединяет деревья в другиъ точках, то с каждым сдвигом от центра $d^{\textbf{'}}$ будем увеличиваться на 2. Теперь найдем минимально возможный $d^{\textbf{'}}$. Он будет равен $d^{\textbf{'}} = \frac{d}{2} + \frac{d}{2}+ 1$, (1 -- длина ребра). Отсюда получаем, что $d^{\textbf{'}} = d + 1 > d$. Итак, нельзя добавить ребро между двумя дереьвями с одинаковыми диаметрами так, чтобы диаметр нового дерева равнялся старым.

\begin{flushright}
\begin{large}
\textbf {нельзя }
\end{large}
\end{flushright}

{\bf 5.} Докажите, что если степень каждой вершины графа не превосходит
$d$, то его можно правильно раскрасить в $d + 1$ цвет.
\begin{center}
\bfseries
{\Large Решение: }
\end{center}

Будем раскрашивать вершины в различные цвета и в произвольном порядке.
На цвет каждой очередной вершины имеется не более $d$ запретов, поэтому
мы сможем ее окрасить в $d+1$ цвет.\\

\begin{flushright}
\begin{large}
\textbf {доказано}
\end{large}
\end{flushright}

{\bf 6.} Назовем не 2-раскрашиваемый граф минимальным, если после уда-
ления любого ребра он становится 2-раскрашиваемым. Докажите, что
в минимальном не 2-раскрашиваемом графе на 1000 вершинах есть
хотя бы одна изолированная вершина (т. е. вершина степени 0).
\begin{center}
\bfseries
{\Large Решение: }
\end{center}

По лемме граф не $2$-раскрашиваемый, если в нем содержиться цикл нечетной длины. Так как по условию минимально не 2-раскрашиваемого графа удаление любого ребра ведет к тому, что граф становиться 2-раскрашиваемым, а значит и исчезает цикл нечетной длины, получаем, что любое ребро -- это часть нечетного цикла, а значит и весь граф -- это нечетный цикл, значит в нём содержится нечётное количество вершин, максимальное количество вершин в графе, состоящих в цикле -- $999$, значит существует как минимум $1$ изолированная вершина.\\ 

\begin{flushright}
\begin{large}
\textbf {доказано }
\end{large}
\end{flushright}

{\bf 7.} Пусть $G$ — связный граф, который не является графом-путём и
$|V (G)| > 3$. Докажите, что в $G$ есть три вершины $v_1, v_2, v_3$, в результате
удаления которых вместе со всеми смежными рёбрами, получается
связный граф $G^{\textbf{'}} = G [V \setminus \{v_1,v_2,v_3 \} ]$.
\begin{center}
\bfseries
{\Large Решение: }
\end{center}

Рассмотрим любой связный граф. Легко показать, что у любого связного графа есть остовное дерево:


Докажем индукцией по числу ребер $n$ ($n \geqslant 0$).

База индукции $n = 0$. Очевидно, получаем множества не связных подграфов графа $G$ , состоящих просто из $1$ вершины. А так как одна вершина -- это дерево, то все выполняется.

Пусть при $n=k$ выполняется. Докажем для $n = k + 1$.

Если в графе $G$ есть ребро, после удаления которого граф $G$ остается связным и, получается, имеет кол-во ребер $|E| = n = k$,  что по предположению индукции верно, тогда имеем граф $G(|E| = k+1)$, который имеет остовое дерево и подходит для $G$.

Если такого ребра нет, то получаем после удаления несвязный граф и, соответсвенно, будет остовное дерево, так как несвязный граф состоит из изолированных вершин или связных графов, тогда остовное дерево есть по индукции или по определению в изолированной вершине.

В данном дереве есть хотя бы одна вершина степени 1 и мы можем удалить ее, а остальные вершины остануться связными, а следовательно и граф останется связным.

Таким образом в любом связном графе можно найти остовное дерево и такую вершину и, соответсвтенно мы можем проделать такую операцию $3$ раза (после каждого раза у нас будет связный граф), получим $G^{\textbf{'}} = G[V\setminus \{ v_1,v_2,v_3\} ]$.

\begin{flushright}
\begin{large}
\textbf {доказано }
\end{large}
\end{flushright}

{\bf 8.} Граф получен из графа-цикла $C_{2n}$, $n > 2$ добавлением рёбер, соеди-
няющих противоположные вершины ($v_1$ соединена с $v_{n+1}, v_2$ с $v_{n+2}$ и т.д.).При каких $n$ получившийся граф правильно раскрашиваемый

{\bf a)} в два цвета; {\bf б)} в три цвета?
\begin{center}
\bfseries
{\Large Решение: }
\end{center}

{\bf a)} Так как цикл чётной длины является 2-раскрашиваемым, тут же возникает необходимое условие: $V_1$ и $V_{n + 1}$ разного цвета, значит 1 и $n+1$ должны быть разной чётности $ \Rightarrow n$ должно быть нечётным.

{\bf б)} Так как при $n$ - нечётном можно раскрасить в $2$ цвета, то и в тоже $3$ можно. Рассмотрим $n$ - чётное.

Тогда можно раскрасить следующим образом : $A_{2k+1}$ -- в один цвет, где $2k+1 < n+1$; $A_{2m}$ -- в другой, где $m < n+1$, а элементы $A_{2n}$ и $A_{n+1}$ в третий цвет. Тогда у нас получится правильная окраска. Сразу стоит отметить ограничение на $k$, поэтому $n > 2$. 

\begin{flushright}
\begin{large}
\textbf {Ответ: {\bf a)} $n$ -- нечетное.\\
{\bf б)} любое $n > 2$.}
\end{large}
\end{flushright}

{\bf 9.} В графе на $100$ вершинах, каждая из которых имеет степень $3$, есть
ровно $600$ путей длины $3$. Сколько в этом графе циклов длины $3$?
\begin{center}
\bfseries
{\Large Решение: }
\end{center}

Так как путь не должен проходить по одному и тому же ребру $2$ раза, то такой путь имеет вид $E_1E_2E_3$. Осталось посчитать, сколькими способами мы можем выбрать эти ребра.

Сначала выбираем ребро $E_2$, оно определяется началом и концом. Начало -- любая вершина; способов выбрать -- $100$ . Конец -- любая из трёх соседних вершин. Итого $300$ способов выбрать $E_2 = \{ u ; v\}$.

Ребро $E_1$ можно выбрать $4$ способами: $2$ способа со стороны вершины $u$ и $2$ способа со стороны вершины $v$. Итого $1200$ способов выбора пути из трёх рёбер.

Так как по условию путей $600$, то остальные $600$ -- это построенные нашими рассуждениями циклы (больше вариантов построения графа на $3$ рёбрах нет). Но нужно учесть, что нам не важен порядок выбора вершин при построении цикла, то есть мы посчитали в $3!$ раз больше, чем реализуется на самом деле. Значит получаем $100$ различных циклов.


\begin{flushright}
\begin{large}
\textbf {Ответ: 100 циклов }
\end{large}
\end{flushright}

{\bf 10.} Докажите, что если размер максимальной клики в графе четный,
то можно раскрасить вершины графа в два цвета так, что размеры
максимальных клик в подграфах обоих цветов равны (подграф инду-
цирован множеством вершин одного цвета).
\begin{center}
\bfseries
{\Large Решение: }
\end{center}


Пусть у нас в общем случае имеется несколько клик максимального размера в графе. Заметим, что у нас не может быть такого, что объединение каких-то двух клик максимального размера в графе -- клика большего размера, так как мы уже рассматриваем максимальные по размеру клики.

Рассмотрим случай, когда у клик максимального размера $n$ в графе чётное количество вершин. Тогда раскрасим эти клики так, чтобы $\frac{n}{2}$ вершин были раскрашены в первый цвет, а остальные вершины -- во второй, тогда у нас в подграфах обоих цветов должны быть одинаковые размеры максимальных клик $\frac{n}{2}$. Для этого вершины остальных клик в графе до раскраски надо раскрасить следующим образом: если число вершин чётное, то раскрасим эти клики так же, как мы уже раскрасили клики максимальных размеров, а если число вершин нечётное и равно $2k + 1$, то $k$ вершин раскрашиваем в первый цвет и $k+1$ во второй. В этом случае получаем, что $k+1 \leqslant \frac{n}{2}$ и $k < \frac{n}{2}$ -- в любом случае получается клика по размеру, не превосходящему $\frac{n}{2}$. 

Все остальные вершины раскрашиваем произвольным образом, так как они не входят в клики и после раскраски тоже не будут входить в клики.

Заметим, что если в графе пересекаются клики максимальных размеров, то мы всё равно можем их раскрасить описанным выше способом: если рассмотреть только 1-ую клику, забыв, что она имеет общие вершины с другой, то в 1-ой клике после раскраски всегда получаем две клики в обоих подграфах размерами $\frac{n}{2}$ (описано выше), аналогично из 2-ой клики можно получить две клики в обоих подграфах размерами $\frac{n}{2}$.

Если же в графе и вовсе одна клика максимального размера, то тогда не возникает проблем с пересечением клик.

Если есть пересечение клик максимального и не максимального размеров, то применим описанные ранее рассуждения для пересечения двух клик максимального размера, за исключением того, что клика меньшего размера разбивается после раскраски на клики размеров $k+1 \leqslant \frac{n}{2}$ и $k < \frac{n}{2}$ или $k < \frac{n}{2}$ и $k < \frac{n}{2}$, где $2k+$ или $2k$ -- размеры клики меньшего размера. Тогда и в этом случае получаем две клики максимального размера в обоих подграфах.

\begin{flushright}
\begin{large}
\textbf {Доказано}
\end{large}
\end{flushright}

\end{document}


