
\documentclass[a4paper,12pt]{article} % тип документа


% Русский язык
\usepackage[T2A]{fontenc} % кодировка
\usepackage[utf8]{inputenc} % кодировка исходного текста
\usepackage[english,russian]{babel} % локализация и переносы


% Математика
\usepackage{amsmath,amsfonts,amssymb,amsthm,mathtools}


\usepackage{wasysym}

%Заговолок
\author{Талашкевич Даниил Александрович}

\title{Бонусная задача № 3 по дискретному анализу}

\date{\today}

\begin{document}

\maketitle
\thispagestyle{empty}

\newpage
\setcounter{page}{1}
\begin{center}
\itshape
\bfseries
{ \Large Problems:}
\end{center}

{\bf Бонусная № 3. }Докажите, что каждое число $\sqrt{a_{1}}+\sqrt{a_{2}}+\cdots+\sqrt{a_{n}}$, где $a_{i} \in \mathbb{N}$ не квадраты целых чисел, иррационально.

\begin{center}
\bfseries
{\Large Решение: }
\end{center}
Докажем по методу математической индукции.\\
\begin{proof}
База индукции ( $n = 1$ ):
Покажем, что корень из любого $p \in \mathbb{N}$, где $p$ не квадрат целого числа -- это иррациональное число:\\
От противного: положим,что $\sqrt{p}$ рационально, тогда представим его в виде $\frac{a}{b}$ (где $\frac{a}{b}$ -- несократимая дробь) $\Rightarrow p = \frac{a^2}{b^2} \Rightarrow p\cdot b^2 = a^2 \Rightarrow $ что $a^2$ делится на $p$, а так как $a$ представляет собой целое число по определению несократимой дроби, то его можно представить в виде $a = f_{1} f_{2}  \dots  f_{n}$,а так как $p \in \mathbb{N}$, то какое-то $f_{i} = p$, тогда у $a^2$ будет множитель $p^2 \Rightarrow a^2$ делится на $p^2$.$a$ представим в виде $kp$, тогда $b^2p = k^2p^2 \Rightarrow b^2=k^2p$. Проведя аналогичные действия имееем,что и $b$ делится на $p$, тогда  представляя его в виде $ b = mp$ получаем, что:
\end{proof}
$ \sqrt{p} = \frac{a}{b} = \frac{kp}{mp}$ -- сократимая дробь, противоречие. Значит мы доказали, что корень из $p \in \mathbb{N}$, где $p$ не квадрат целого числа -- это иррациональное число.\\
Пусть при $n = k$ : $\sum\limits_{i=1}^{n} \sqrt{a_{i}}$ -- иррационально.\\
Тогда докажем для $n = k + 1$ :
\begin{proof}
1.Обозначив $\sum\limits_{i=1}^{k} \sqrt{a_{i}}$ за $P$, где $P$ -- иррациональное число. Тогда $\sum\limits_{i=1}^{k+1} \sqrt{a_{i}} = P + \sqrt{a_{k+1}}$. Обозначим $\sqrt{a_{k+1}}$ за $\sqrt{Q}$ для удобства.\\
Докажем, что $ R = P + \sqrt{Q}$ -- иррациональное. Пойдем методом от противного :

$R$ -- рациональное, тогда и $R^2$ -- рациональное : $R^2 = P^2 + Q + 2P\cdot \sqrt{Q}.$ $Q \in \mathbb{N}$, тогда остается рассмотреть $P^2 + 2P\cdot \sqrt{Q}$ :\\
Обозначим $P^2 + 2P\cdot \sqrt{Q}$ за $R_1$. Опять пойдем от противного : пусть $R_1$ рациональное, тогда :

$R_1= P(P+2\sqrt{Q})$, где $P$ иррациональное по определению, тогда рассмотрим $(P+2\sqrt{Q})$.От противного : пусть $R_2 = (P+2\sqrt{Q})$ -- рациональное, тогда $R_2^2$ -- тоже.
$R_2^2 = P^2 + 4Q + 4P\sqrt{Q}$. $4Q$ не иррациональное. Теперь докажем, что при наших условиях на $P$ и $Q$ будет выпоняться, что $4P\sqrt{Q}$ -- иррациональное.

$P\sqrt{Q}$ может быть рациональным, только если $P$ может быть представленно в двух видах: 1) $P = k\sqrt{Q}$  или 2) $P = \frac{k}{\sqrt{Q}}$ (где $k$ рациональное число). Рассмотрим оба случая:\\

1)Если $P = k\sqrt{Q}$, то $P = \sqrt{k^2\cdot Q}$ (обозначим $P_1 = k^2\cdot Q$ -- не полный квадрат натурального числа). Тогда докажем, что сумма 2 корней из неполных квадратов - это иррациональное число.\\
Тогда рассмотрим уже искомую $R = \sqrt{P_1} + \sqrt{Q}$. $R$ иррационально, если ${P_1}$ и $ Q$ не являются квадратами рациональных чисел. Предположим противное; т.е. что $R$ рационально. Тогда возведение в квадрат обеих частей уравнения $R = \sqrt{P_1} + \sqrt{Q}$ дает

\[ R^2 = P_1 + 2\cdot \sqrt{{P_1}Q} + Q.\] и поэтому \[\sqrt{{P_1}Q} = (R^2 - {P_1} - Q) \slash 2.\]
Правая часть обязательно является рациональным числом. Если $({P_1}Q)$ не является квадратом рационального числа, то противоречие.\\
Однако, даже если $({P_1}Q)$ - квадрат рационального числа, $R$ может не быть рациональным. Предположим, что $Q$ равно ${P_1}$. Тогда $R = 2\sqrt{{P_1}}$. Если ${P_1}$ не является квадратом рационального числа ( а оно не является), то $R$ обязательно иррационально.\\
Предположим, что $({P_1}Q)$ квадрат рационального числа, но ${P_1}$ не квадрат рационального числа и , пусть без ограничения общности положим, ${P_1} < Q$. Это означает, что для того, чтобы $(PQ)$ было квадратом рационального числа $Q = {P_1}S^2$,где $S$ -- рациональное число. Таким образом, $ R = \sqrt{{P_1}} (1 + S)$. Это означает, что $R$ является произведением иррационального числа и рационального числа и, следовательно, иррационально.\\

2)$P = \frac{k}{\sqrt{Q}}$. $P = \frac{k\cdot \sqrt{Q}}{Q}$ и можно свести к п.1.Или же искомия $R = P + \sqrt{Q} = \frac{k}{\sqrt{Q}}+ \sqrt{Q} = \frac{k + Q}{\sqrt{Q}}$, где $k+Q$ - не иррационально, а $\sqrt{Q}$ иррационально.\\
Таким образом, для всех случаев, кроме $a_i \neq N^2$ ($N \in \mathbb{N}$), $R = \sqrt{{P_1}} + \sqrt{Q}$ иррационально.
Мы доказали,  что  $\sum\limits_{i=1}^k a_{i} + a_{k+1}$ -- иррационально.
\end{proof}
Доказали методом математической индукции.
\begin{flushright}
\begin{large}
\textbf {Ответ: доказано. }
\end{large}
\end{flushright}







\end{document}