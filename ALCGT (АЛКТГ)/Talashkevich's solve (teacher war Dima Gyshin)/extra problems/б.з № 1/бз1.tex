
\documentclass[a4paper,12pt]{article} % тип документа


% Русский язык
\usepackage[T2A]{fontenc} % кодировка
\usepackage[utf8]{inputenc} % кодировка исходного текста
\usepackage[english,russian]{babel} % локализация и переносы


% Математика
\usepackage{amsmath,amsfonts,amssymb,amsthm,mathtools}


\usepackage{wasysym}

%Заговолок
\author{Талашкевич Даниил Александрович}

\title{Бонусная задача №1 по дискретному анализу}

\date{\today}

\begin{document}

\maketitle
\thispagestyle{empty}

\newpage
\setcounter{page}{1}
\begin{center}
\itshape
\bfseries
{ \Large Problems:}
\end{center}

{\bf Бонусная задача.}  Найдите асимптотическую оценку количества функций от $n$ переменных, которые зависят от всех своих аргументов существенно. Иначе говоря, надо придумать такие верхнюю и нижнюю оценки на это количество, чтобы их отношение стремилось к $1$ при $n\rightarrow \infty$.


\begin{center}
{\Large Решение: }
\end{center}
Посчитаем общее количество всех булевых функций от n переменных. Всего разных наборов из $n$ нулей и единиц будет $2^n$ штук, т.к. для каждой из $n$ переменных будет 2 варианта. Иными словами, таблица истинности содержит $2^n$ строк. Поскольку на каждом из $2^n$ наборов функция может принимать одно из двух значений, то общее количество булевых функций составит $2^{2^n}$.\\
И так имеет, что общее число булевских функций равно $2^{2^n}$.\\
По условию задачи у всех этих функций все $n$ переменных будут существенными. Если предположить, что $n_i$ переменная фиктивна, то ее можно удалить и получаем б.ф. от $n-1$ переменной. Тогда отношение "неправильных" функций к общему количеству мало, то есть

$\frac{y(n)}{2^{2^n}} \rightarrow 0$ , при $n \rightarrow \infty$. Значит отношение "правильных" функций к общему кол-ву:

$\frac{x(n)}{2^{2^n}} \rightarrow 1$ , при $n \rightarrow \infty$.
Ввиду того, что $x(n) + y(n) = 2^{2^n}$. Получаем асимптотически точную оценку. Она заключена между $  2^{2^n}-n2^{2^{n-1}}$ и $2^{2^n}.$
\begin{flushright}
\begin{large}
\textbf {Ответ: между $  2^{2^n}-n2^{2^{n-1}}$ и $2^{2^n}.$ }
\end{large}
\end{flushright}


\end{document}
